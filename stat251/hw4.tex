\documentclass{article}
\usepackage{geometry}
\usepackage[namelimits,sumlimits]{amsmath}
\usepackage{amssymb,amsfonts}
\usepackage{multicol}
\usepackage{graphicx}
\usepackage[cm]{fullpage}
\newcommand{\tab}{\hspace*{5em}}
\newcommand{\conj}{\overline}
\newcommand{\dd}{\partial}
\newcommand{\ep}{\epsilon}
\newcommand{\openm}{\begin{pmatrix}}
\newcommand{\closem}{\end{pmatrix}}
\DeclareMathOperator{\cov}{cov}
\DeclareMathOperator{\var}{var}
\newcommand{\nc}{\newcommand}
\newcommand{\rn}{\mathbb{R}}
\nc{\Pt}[1]{P(\text{#1})}
\nc{\nn}{\mathbb{N}}
\begin{document}
Name: Hall Liu

Date: \today 
\vspace{1.5cm}

\subsection*{1}
Using the definition of density, the first term is equal to $\int_0^\infty\int_x^\infty f(t)dtdx$. The region we're integration over is the part of the first quadrant above the diagonal (with $t$ on vertical and $x$ on horizontal), so we can re-order the integration as
$$\int_0^\infty\int_0^tf(t)dxdt=\int_0^\infty tf(t)dt$$
Similarly, the second term is 
$$\int_0^\infty\int_{-\infty}^{-x}f(t)dtdx=\int_{-\infty}^0\int_0^tf(t)dtdx=\int_{-\infty}^0tf(t)dt$$
Adding the two gives the definition of expectation.
\subsection*{2}
Rewriting $P(X>x)$ as $\int_x^\infty f(t)dt$ as above and using the same re-ordering, we have
$$\int_0^\infty nx^{n-1}P(X>x)dx=\int_0^\infty\int_x^\infty nx^{n-1}f(t)dtdx=\int_0^\infty\int_0^tnx^{n-1}f(t)dxdt=\int_0^\infty t^nf(t)dt=E(X^n)$$
\subsection*{3}
We have $\var(X)=\var(X-a/2)=E((X-a/2)^2)-(E(X-a/2))^2$. Since $X-a/2\in[-a/2,a/2]$, $(X-a/2)^2\leq a^2/4$, which means that the expectation is $\leq a^2/4$. Since the second term is nonnegative, we have that $\var(X)\leq a^2/4$.
\subsection*{4}
I think this question is asking for $\text{ceil}(X)$? This is a discrete RV with pmf $p$, where $p(n)=P(n-1<X\leq n)=\int_{n-1}^ne^{-x}dx=e^{-(n-1)}-e^{-n}$.
\subsection*{5}
As $n\to\infty$, $X_n-n/2$ converges in distribution to a normal RV $N(0,n/4)$ by the central limit theorem. Thus, we want to find 
$$\lim_{n\to\infty}\sqrt{n}\int_{x\sqrt{n}}^{x\sqrt{n}+1}\frac{1}{\sqrt{n\pi/2}}e^{-2t^2/n}=\frac{1}{\sqrt{\pi/2}}\lim_{n\to\infty}\int_{x\sqrt{n}}^{x\sqrt{n}+1}e^{-2t^2/n}$$
The latter integral is bounded between $e^{-2x^2}$ and $e^{(-2(x\sqrt{n}+1)^2)/n}$ by evaluating at the lower and upper bounds and using the monotony of the integrand. Expanding the lower bound, we have $e^{-2x^2}e^{-4x/\sqrt{n}}e^{-2/n}$. As $n\to\infty$, the second two exponential terms approach $1$, so the lower bound converges to the upper bound as $n\to\infty$, and thus we have that the limit is $\frac{1}{\sqrt{\pi/2}}e^{-2x^2}$.
\subsection*{6}
Since $F$ is order-preserving, we have $F^{-1}(U)\leq y\Longleftrightarrow U\leq F(y)$. Thus, $P(F^{-1}(U)\leq y)=P(U\leq F(y))$. Since $U$ is uniform, this probability is just $F(y)$.
\subsection*{7}
Take the integral $\frac{1}{\sqrt{2\pi}}\int_{-\infty}^\infty|x|e^{-x^2/2}$ as the expected value of $|X|$. We can split this integral into two parts, the positive and negative, so we get
$$\frac{1}{\sqrt{2\pi}}\left(\int_0^\infty xe^{x^2/2}-\int_{-\infty}^0xe^{-x^2/2}\right)=\frac{1}{\sqrt{2\pi}}\left(\int_{-\infty}^0e^u-\int_0^{-\infty}e^u\right)=\frac{2}{\sqrt{2\pi}}$$
\subsection*{8}
We have $P(U\leq x)=x$ for $x\in[0,1]$, so $P(1-U\leq 1-x)=x\implies P(1-U\geq y)=1-y\implies P(1-U\leq y)=y$, so $1-U$ is also uniform.

$P(\max(U,1-U)\leq x)=0$ for $x\leq0.5$, so this is a distribution on $[0.5,1]$. If we want $\max(U,1-U)\leq x$, we need to have $U\leq x$ and $1-U\leq x\implies U\geq 1-x$. Thus, the probability is $x-(1-x)=2x-1$, which means that the density is $2$, so it's uniform on $[0.5,1]$
\subsection*{9}
Let $g_n(x)$ be the rational approximation function of order $n$. Then, $g_n(X)$ and $g_n(Y)$ are discrete random variables, with $P(g_n(X)=x)=\int_{k/n}^{(k+1)/n}f_X(x)$. The cdf of $g_n(X)$ converges to that of $X$ as $n\to\infty$. To see this, note that $P(g_n(X)\geq x)=\int_a^\infty f_X(x)$, where $|a-x|<\frac{1}{n}$ due to the fineness of the rational approximation. Thus, taking $n\to\infty$ brings this to $P(X\geq x)$. Now, since we know that expectations of discrete RVs are linear, we have $E(g_n(X)+g_n(Y))=E(g_n(X))+E(g_n(Y))$ for all $n$, so taking $n\to\infty$ here gives the linearity for continuous RVs as we wanted.
\end{document}
