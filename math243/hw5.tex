\documentclass{article}
\usepackage{geometry}
\usepackage[namelimits,sumlimits]{amsmath}
\usepackage{amssymb,amsfonts}
\usepackage{multicol}
\usepackage{mathrsfs}
\usepackage[cm]{fullpage}
\newcommand{\nc}{\newcommand}
\newcommand{\tab}{\hspace*{5em}}
\newcommand{\conj}{\overline}
\newcommand{\dd}{\partial}
\nc{\cn}{\mathbb{C}}
\nc{\rn}{\mathbb{R}}
\nc{\qn}{\mathbb{Q}}
\nc{\zn}{\mathbb{Z}}
\nc{\aff}{\mathbb{A}}
\nc{\proj}{\mathbb{P}}
\nc{\pd}[2]{\frac{\partial {#1}}{\partial {#2}}}
\nc{\ep}{\epsilon}
\nc{\topo}{\mathscr{T}}
\nc{\basis}{\mathscr{B}}
\nc{\nullset}{\varnothing}
\nc{\openm}{\begin{pmatrix}}
\nc{\closem}{\end{pmatrix}}
\begin{document}
Name: Hall Liu

Date: \today 
\vspace{1.5cm}
\subsection*{5.1.1}
Lines in $\proj^2$ are of the form $V(ax+by+cz)$. If $c\neq0$, we essentially have $z=ax+by$, so the image of this under the Veronese map $v_2$ consists of the points $[x^2:y^2:(ax+by)^2:xy:x(ax+by):y(ax+by)]$, where $x$ and $y$ are not both $0$. If $c=0$, then we can just take $a$ or $b$ nonzero and get essentially the same result, just with the coordinates shifted around. If we let $x^2=z_1,xy=z_2$, and $y^2=z^3$, we have that the image consists of points like $[z_1:z_2:az_1+2abz_3+bz_2:z_3:az_1+bz_3:az_2+bz_3]$. If we let the $i$th coordinate be $w_i$, then this satisfies the linear constraints $w_3-aw_1-2abw_4-bw_2,w_5-aw_1-bw_3$, and $w_6-aw_2-bw_4$.
\subsection*{5.2.1}
First, we do the degenerate case. If a conic determined by $5$ points is degenerate, it is either a double line or a line pair, since the others contain less than $5$ distinct points. In the case of a double line, all $5$ points are collinear, so we have that three points are collinear. If the conic is a line pair, then each point must be on one of the lines, and by the Pidgeonhole principle, one line must contain at least three points, so those three points are collinear.

Now, suppose that there does not exist a unique conic. Let two distinct conics passing through $p_1,\ldots,p_5$ be $C_1,C_2$. If both $C_1$ and $C_2$ are non-degenerate, then they're both irreducible, and Bezout's theorem implies that they intersect in at most $4$ points, which is a contradiction. Thus both of $C_1$ or $C_2$ must be degenerate, and they must have a common component (a line). If either of $C_1$ or $C_2$ is the double line, we have that all $5$ points are collinear, so we're down to the case of $C_1$ and $C_2$ both being line pairs. Let $C_1=L_0\cup L_1$ and $C_2=L_0\cup L_2$. If $L_0$ contains less than $4$ points, $L_1$ must contain more than one point not in $L_0$, but this implies that $L_2$ contains the same two points, making $C_1=C_2$. Thus the $4$ points on $L_0$ must be collinear.
\subsection*{5.2.2}
By Thm from class, all cubics are equivalent to $\ep_1x^2+\ep_2y^2+\ep_3z^2$ after a linear change of coordinates. 
\subsection*{5.3.2}
Suppose $X\subset \proj^m$ is defined by $Z(f_1,\ldots,f_r)$ where $f_i\in k[x_0,\ldots,x_m]$ and $Y\subset \proj^n$ is defined by $Z(g_1,\ldots,g_s)$ where $g_i\in k[x_{m+1},\ldots,x_{m+n+1}]$. Then, looking at $f_i$ and $g_i$ as elements of $k[x_0,\ldots,x_{m+n+1}]$, all the points in $X\times Y\subset \proj^m\times\proj^n$ are defined by the $f_i$ and the $g_i$. Now, if we take the image of $X\times Y$ under the Segre embedding, the image will satisfy the transformed versions of the $f_i$ and $g_i$ under the Segre embedding as well as the equations defining the image of $\proj^m\times\proj^n$. Since the Segre embedding is an isomorphism onto the image, the variety defined by these equations is in fact the image of $X\times Y$.

As for quasiprojective varieties, they are defined in $\proj^n,\proj^m$ by the intersection of a variety (e.g. $X,Y$) with the complement of another variety ($X',Y'$), so we can map these varieties in the same way as above.
\end{document}
