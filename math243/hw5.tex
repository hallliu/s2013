\documentclass{article}
\usepackage{geometry}
\usepackage[namelimits,sumlimits]{amsmath}
\usepackage{amssymb,amsfonts}
\usepackage{multicol}
\usepackage{mathrsfs}
\usepackage[cm]{fullpage}
\newcommand{\nc}{\newcommand}
\newcommand{\tab}{\hspace*{5em}}
\newcommand{\conj}{\overline}
\newcommand{\dd}{\partial}
\nc{\cn}{\mathbb{C}}
\nc{\rn}{\mathbb{R}}
\nc{\qn}{\mathbb{Q}}
\nc{\zn}{\mathbb{Z}}
\nc{\aff}{\mathbb{A}}
\nc{\proj}{\mathbb{P}}
\nc{\pd}[2]{\frac{\partial {#1}}{\partial {#2}}}
\nc{\ep}{\epsilon}
\nc{\topo}{\mathscr{T}}
\nc{\basis}{\mathscr{B}}
\nc{\nullset}{\varnothing}
\nc{\openm}{\begin{pmatrix}}
\nc{\closem}{\end{pmatrix}}
\begin{document}
Name: Hall Liu

Date: \today 
\vspace{1.5cm}
\subsection*{5.1.1}
Lines in $\proj^2$ are of the form $V(ax+by+cz)$. If $c\neq0$, we essentially have $z=ax+by$, so the image of this under the Veronese map $v_2$ consists of the points $[x^2:y^2:(ax+by)^2:xy:x(ax+by):y(ax+by)]$, where $x$ and $y$ are not both $0$. If $c=0$, then we can just take $a$ or $b$ nonzero and get essentially the same result, just with the coordinates shifted around. If we let $x^2=z_1,xy=z_2$, and $y^2=z^3$, we have that the image consists of points like $[z_1:z_2:az_1+2abz_3+bz_2:z_3:az_1+bz_3:az_2+bz_3]$. If we let the $i$th coordinate be $w_i$, then this satisfies the additional linear constraints $w_3-aw_1-2abw_4-bw_2,w_5-aw_1-bw_3$, and $w_6-aw_2-bw_4$.
\subsection*{5.2.1}
First, we do the degenerate case. If a conic determined by $5$ points is degenerate, it is either a double line or a line pair, since the others contain less than $5$ distinct points. In the case of a double line, all $5$ points are collinear, so we have that three points are collinear. If the conic is a line pair, then each point must be on one of the lines, and by the Pigeonhole principle, one line must contain at least three points, so those three points are collinear.

Now, suppose that there does not exist a unique conic. Let two distinct conics passing through $p_1,\ldots,p_5$ be $C_1,C_2$. If both $C_1$ and $C_2$ are non-degenerate, then they're both irreducible, and Bezout's theorem implies that they intersect in at most $4$ points, which is a contradiction. Thus both of $C_1$ or $C_2$ must be degenerate, and they must have a common component (a line). If either of $C_1$ or $C_2$ is the double line, we have that all $5$ points are collinear, so we're down to the case of $C_1$ and $C_2$ both being line pairs. Let $C_1=L_0\cup L_1$ and $C_2=L_0\cup L_2$. If $L_0$ contains less than $4$ points, $L_1$ must contain more than one point not in $L_0$, but this implies that $L_2$ contains the same two points, making $C_1=C_2$. Thus the $4$ points on $L_0$ must be collinear.
\subsection*{5.2.2}
We want to show that the degenerate conics are determined by a set of polynomial equations in the coefficients of the conic. Since the equation of a projective conic is in one-to-one correspondence with symmetric bilinear forms, we have that each conic is associated with a $3\times3$ symmetric real matrix $A$. Further, by the classification of projective conics, for each such matrix $A$ there exists a nonsingular matrix $C$ such that $C^TAC=B$, where $B$ has all zeros and ones along the diagonal. The degenerate conics therefore correspond to those $B$ which have a zero on the diagonal or are all $1$ along the diagonal. We then have $\det(A)\det(C)^2=0\implies \det(A)=0$ or $\det(A)\det(C)^2=1$. In the case where $\det(A)=0$, this is a polynomial condition on the coefficients of the conic. In the second case, since we can scale $A$ by any real number, we might as well take $\det(A)=1$, which results in another polynomial condition, so the set of degenerate conics is closed, which means that the nondegenerate conics are open.

A conic is projectively equivalent to a double line iff there exists an element of $PGL_3$ that can transform it into $x^2=0$ when applied to the variables. Equivalently, this is saying that there exists an element of $PGL_3$ that takes $x^2=0$ to the conic, which means that the conic is of the form $(ax+by+cz)^2=a^2x^2+b^2y^2+c^2z^2+abxy+acxz+bcyz$ with $[a:b:c]\in\proj^2$. Note that this exactly the image of the Veronese map $[a:b:c]\mapsto [a^2:b^2:c^2:ab:ac:bc]$.
\subsection*{5.2.3}
Given $4$ points, no three collinear, the space of cubics passing through them is the intersection of $4$ independent hyperplanes in $\proj^5$, which means that it's a line in $\proj^5$. We can parametrize the line in $\proj^5$ with one variable $t$ (keeping the sum of the coordinates fixed at $1$), so all the coefficients of the conics that pass through the $4$ points can be seen as linear functions of $t$. Now, if we restrict all these to some given line in $\proj^2$, we get a quadratic whose discriminant is a quadratic function of $t$, and as long as the discriminant of the discriminant is greater than zero, we get two solutions for $t$ that make the conic tangent to the line. 

We claim that this condition on the discriminant holds iff the given line doesn't pass through any of the $4$ points. If the line in $\proj^2$ doesn't pass through any of the $4$ points, then there will be no point on the line where the restriction vanishes for all $t$. 
\subsection*{5.3.2}
Suppose $X\subset \proj^m$ is defined by $Z(f_1,\ldots,f_r)$ where $f_i\in k[x_0,\ldots,x_m]$ and $Y\subset \proj^n$ is defined by $Z(g_1,\ldots,g_s)$ where $g_i\in k[x_{m+1},\ldots,x_{m+n+1}]$. Then, looking at $f_i$ and $g_i$ as elements of $k[x_0,\ldots,x_{m+n+1}]$, all the points in $X\times Y\subset \proj^m\times\proj^n$ are defined by the $f_i$ and the $g_i$. Now, if we take the image of $X\times Y$ under the Segre embedding, the image will satisfy the transformed versions of the $f_i$ and $g_i$ under the Segre embedding as well as the equations defining the image of $\proj^m\times\proj^n$. Since the Segre embedding is an isomorphism onto the image, the variety defined by these equations is in fact the image of $X\times Y$.

As for quasiprojective varieties, they are defined in $\proj^n,\proj^m$ by the intersection of a variety (e.g. $X,Y$) with the complement of another variety ($X',Y'$), so we can map these varieties in the same way as above.
\subsection*{5.3.4}
%The projections from the product topology to one of the component spaces are open maps. We want to show that there exists some set open in $X\times Y$ whose image under the projection is not open. To do this, we can show that there exists open sets which cover $\proj^n\times\proj^m$ except for a finite number of points and whose images are not open, since then we can just intersect the open set with the variety and get an open set whose image isn't open. First, as an example, let's do it for $\proj^1\times\proj^1$. The Segre variety of this lies in $\proj^3$. Note that the varieties $V(x_1^2-x_2),V(x_1^2-x_3),V(x_2^2-x_4),V(x_3-x_4^2)$ only intersect at the point $[1:1:1:1]$, so their complements cover $\proj^3$ except for that point. 

%The image of the Segre map in $\proj^3$ has the form $[su:sv:tu:tv]$, and this intersected with the complement of $V(x_i^2-x_j)$ (say $i=1,j=2$) will carry the condition that $s^2u^2\neq sv$. Projecting to $\proj^1$, we get $[su:sv]$ with this condition. However, this is the complement of $\{[x:x^2]:x\neq0\}$ in $\proj^1$, which is only the point $[1:0]$, which is not open. This works the same way for the other $i,j$ pairs. 

%Moving into $\proj^n\times\proj^m$, we use the complements of the same varieties (which again only intersect at the point consisting of all $1$s). 
\subsection*{5.4.1}
Let the conic be $x^2+y^2-z^2$, since all the interesting irreducible conics are equivalent to this after a change of coordinates (nothing can meet the double line in two distinct points, so the set of lines would be the whole Grassmannian). Restricting this to a line $ax+by+cz$ (assuming $c\neq0$, which we can do because we can just swap coordinates around again), we have $x^2+y^2-(a^2x^2+abxy+b^2y^2)=(1-a^2)x^2-abxy+(1-b^2)y^2$. This has less than two solutions if the discriminant $a^2b^2-4(1-a^2)(1-b^2)=a^2b^2-4(1-a^2-b^2+a^2b^2)$ is nonpositive. 
\subsection*{3}
An affine conics is just its homogenized projective conic restricted to an affine patch, and we know what projective conics look like, so we can start from there. Apply some projective linear transformation to bring it into one of the standard forms, and consider the intersection of its affine cone with a plane not passing through the origin. For simplicity's sake, I'm doing this qualitatively, since it's going to be ugly if I start doing this case-by-case with symbols.

Assume the projective conic is non-degenerate. Then the affine cone is a cone coming out of the origin in the direction of the $z$-axis. If we take a plane inclined at no more than the slopes of the cone, we get a ellipse. If we take a plane inclined at the angle of the slopes of the cone, we get a parabola. If the plane is inclined more than that, it's a hyperbola. 

If the projective conic is empty, any of its affine sections are going to be empty. 

If the projective conic is a pair of lines, the affine cone is the intersection of the $x=y$ plane and the $x=-y$ plane. If the cut-plane is not orthogonal to the $x-y$ plane, then it will include a point on the $z$-axis and present as a pair of intersecting lines. If the cut-plane is orthogonal and is parallel to one of the planes, it will appear as a line, and if it is orthogonal but not parallel to either plane, it will intersect both planes and appear as a pair of parallel lines.

If the projective conic is a point, its affine cone is the $z$-axis, and the only way a plane can intersect that is at a point.

If the projective conic is a double line, the affine cone is a ``double plane'' , and the cut-plane intersects that at a double line.
\subsection*{4}
a. We can apply a projective change of coordinates to bring the three disjoint lines $L,M,N$ onto the Segre variety as the lines $[u:v:0:0],[0:0:u:v],[u:v:u:v]$ resp., parametrized by points $[u:v]$ in $\proj^1$. If we then fix $[u:v]$ in $\proj^1$, we get lines $[su:sv:tu:tv]$ which intersect the transformed versions of $L,M,N$ at $[s:t]=[1:0],[0:1],[1:1]$, resp. This means that the lines which intersect $L,M,N$ cover the Segre variety. Now we want to show that any line which intersects $L,M,N$ must lie on the Segre variety. Any such line $ax+by+cz+dw=0$ intersects the first two lines at $[u_1:v_1:0:0]$ and $[0:0:u_2:v_2]$, so we have $au_1+bv_1=0$ and $cu_2+dv_2=0$. If we want to intersect $N$, we need to find some $[u:v]$ such that $(a+c)u+(b+d)v=0$, which is impossible unless $[u_1:v_1]=[u_2:v_2]$, which means that the line $ax+by+cz+dw=0$ is in fact of the form $[su:sv:tu:tv]$. I'm not sure what ``unique Segre variety'' means. Does it mean the unique Segre image of some product of subvarieties $U\times V$? It's not true then, since $\proj^1\times\{[1:0],[0:1],[1:1]\}$ is just the three lines (after a transformation).

b. The Segre variety $\Sigma_{k-1,1}$ is parametrized as $[a_1u:\cdots:a_{k}u:a_1v:\cdots:a_kv]$. As above, we can apply some projective change of coordinates to bring the three $k-1$-planes to $[a_1:\cdots:a_k:0:\cdots:0], [0:\cdots:0:a_1:\cdots:a_k],[a_1:\cdots:a_k:a_1:\cdots:a_k]$. For each $[b_1:\cdots:b_k]\in\proj^{k-1}$, we have a line $[b_1u:\cdots:b_{k}u:b_1v:\cdots:b_kv]\in\proj^{2k-1}$ parametrized by $[u:v]\in\proj^1$, and they each intersect the three planes at $[u:v]=[1:0],[0:1],[1:1]$, resp. Since these account for all the points in the Segre variety, the union of this family of lines covers the Segre variety. Conversely, given any line through these three planes, it must pass through $[x_1:x_2:\cdots:x_k:0:\cdots:0]$ and $[0:\cdots:0:x_{k+1}:\cdots:x_{2k}]$ for some $x_i,i\in[1,2k]$. We can then parametrize this line as $[x_1u:x_2u:\cdots:x_ku:-x_{k+1}v:\cdots:-x_{2k}v]$ by $[u:v]$. If we want this to pass through the third plane, we'll need $[u:v]$ such that the first $k$ coordinates are the same as the second two, but this can only happen if the $x_i=\lambda x_{i+k}$ for $i\leq k$. Change the parametrization so that we're doing it by $[u:\lambda v]$, and we end up with a line of the form described above that's in the Segre variety. As before, I have no idea what unique means here.

\subsection*{5}
Lines in $\proj^5$ are identified with elements of Gr$(2,6)$. They have $15$ affine coordinates and lie in $\proj^{14}$. There's some polynomial constraint on these coordinates that makes the lines in $\Sigma_{2,1}$. 
\subsection*{6}
a. The diagonal has the form $[x_0:\cdots:x_n]\times[x_0:\cdots:x_n]$, so the image under the Segre map is parametrized by the $x_i$ as $[x_0^2:x_0x_1:\cdots:x_1x_0:x_1^2:\cdots:x_n^2]$, or in other words, some arrangement of all the degree 2 monomials in $x_0,\ldots,x_n$. If we look at the map $\proj^n\to\proj^n\times\proj^n\to\Sigma_{n,n}$, $x\to(x,x)\to\sigma(x)$, this is exactly the Veronese map $v_2$ from $\proj^n$ to $\proj^{n^2+2n}$. The first part of the map has image $\Delta$, so the second part of the map is the Segre map restricted to $\Delta$, so they have the same image.

b. The diagonal as a projective variety is the image of the diagonal in $X\times X$ under the Segre map, so it's actually the image of $X$ under the Veronese map by the above discussion (replacing $\proj^n$ with $X$). Since the Veronese map is a closed map, the image of the diagonal is a variety.
\end{document}
