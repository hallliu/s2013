\documentclass{article}
\usepackage{geometry}
\usepackage[namelimits,sumlimits]{amsmath}
\usepackage{amssymb,amsfonts}
\usepackage{multicol}
\usepackage{mathrsfs}
\usepackage[cm]{fullpage}
\newcommand{\nc}{\newcommand}
\newcommand{\tab}{\hspace*{5em}}
\newcommand{\conj}{\overline}
\newcommand{\dd}{\partial}
\nc{\cn}{\mathbb{C}}
\nc{\rn}{\mathbb{R}}
\nc{\qn}{\mathbb{Q}}
\nc{\zn}{\mathbb{Z}}
\nc{\aff}{\mathbb{A}}
\nc{\proj}{\mathbb{P}}
\nc{\pd}[2]{\frac{\partial {#1}}{\partial {#2}}}
\nc{\ep}{\epsilon}
\nc{\topo}{\mathscr{T}}
\nc{\basis}{\mathscr{B}}
\nc{\nullset}{\varnothing}
\nc{\openm}{\begin{pmatrix}}
\nc{\closem}{\end{pmatrix}}
\setlength{\parindent}{0mm}
\DeclareMathOperator{\rk}{rank}
\begin{document}
Name: Hall Liu

Date: \today 
\vspace{1.5cm}

\subsection*{3.2.1}
Let $\{U_i\}$ be an open cover of a projective variety $Z$. Consider the preimage of $Z$ in $\aff^{n+1}$ and each of the open covering sets under the quotient map. Consider the intersection of $\pi^{-1}(Z)$ with the set of points with absolute value $1$. This being a closed and bounded set, it is compact, so there exists a finite subcover by preimages of the $U_i$. Then, the finite set of the $U_i$ from this subcover also cover $Z$, since $[z_0:\ldots z_n]\in Z\implies 1/|z|(z_0,\ldots,z_n)\in\pi^{-1}(U_i)$ for some $i$, and taking the image under $\pi$ gives that $[z_0:\ldots:z_n]\in U_i$.

We can find an affine variety embedded inside a projective variety $Z$ by considering the points of $Z$ where $x_0\neq0$. If $Z$ is defined by a set of homogenous polynomials $\{F_i\}$, then the affine variety $V$ is defined by the set of polynomials obtained by evaluating each of the $F_i$ at $x_0=1$. We already have that $Z$ is compact in both the Euclidean and Zariski toplogies (Zariski is coarser than Euclidean), so we need to show that $\bar{V}=Z$ in the Euclidean topology. Consider $Z-V$. This is the set of points where $x_0=0$. Looking at the preimage in $\aff^{n+1}$, the preimage of $Z-V$ is the preimage of $Z$ intersected with a hyperplane, so all such points are limit points of the preimage of $V$, which makes the closure of $V$ $Z$.
\subsection*{3.2.2}
Call the three variables $x,y,$ and $z$. To go from a homogenous polynomial of degree $d$ in 3 variables to a polynomial in 2 variables, evaluate at $z=1$ ($\psi$). To go the other way, multiply each monomial term by a power of $z$ so that it has degree $d$ ($\psi^{-1}$). This is a bijection, as $\psi^{-1}\circ\psi$ is just adding on some number of $z$ terms then taking them away again by making them $1$, so it's the identity. $\psi$ on a monomial $x^ay^bz^c$ ($a+b+c=d$) takes it to $x^ay^b$, then $\psi^{-1}$ takes that to $x^ay^bz^c$ again. 

In the affine patches, the subspace topology has closed sets consisting of $V\cap W'\cap\aff^2$, where $V'$ is some other projective variety. The affine Zariski topology has closed sets consisting of $V\cap\aff^2\cap W$, where $W$ is some affine variety in $\aff^2$. For each affine variety $W$, we can produce a projective variety $W'$ by taking $V(\psi^{-1}(I(W)))$. $W'$ has the same zero set as $W$ when restricted to the affine patch of $\proj^2$, since $f(w_1,w_2,1)=0$ iff $\psi(f)(w_1,w_2)=0$. Thus we have $V\cap\aff^2\cap W=V\cap W'\cap\aff^2$. Similarly, for each projective variety $W'$, we can produce an affine variety $W$ in the same manner, and we have that each closed set of the form $V\cap W'\cap\aff^2$ is equal to one of the form $V\cap\aff^2\cap W$, showing that the topologies are equivalent.
\subsection*{3.3.1}
Suppose we can write $\overline{V}=Y\cup Z$, with both $Y,Z$ being proper subvarieties of $\conj{V}$. Then, we have $V=\conj{V}\cap U_0=(Y\cap U_0)\cup(Z\cap U_0)$. Since $\conj{V}$ is the smallest closed set containing $V$, both $Y-V$ and $Z-V$ must be nonempty, so $Y\cap U_0$ and $Z\cap U_0$ are proper affine subvarieties of $V$, which contradicts $V$'s irreducibility.
\subsection*{3.3.2}
Plugging the parametrization $(t,t^2,t^3)$ into these polynomials yields zero in both cases. Going the other way, $f=y-x^2$ requires that any solution have the form $(x,x^2,z)$, and $g=z^2-2xyz+y^3=0$ requires that $z^2-2x^3z+x^6=0\implies(z-x^3)^2=0\implies z=x^3$, so we have that all solutions are of the form $(x,x^2,x^3)$. When we homogenize, we get the polynomials $\tilde{f}=wy-x^2$ and $\tilde{g}=wz^2-2xyz+y^3$. Since both of these are in $\tilde{I}$, we have that $\conj{V}\subset V(\tilde{f},\tilde{g})$. To show the converse, we want to show that $\tilde{f}$ and $\tilde{g}$ only vanish at points $[0:x:y:z]$ in the set of limit points of $V$ in $\proj^3$. If we set $w=0$, we get that both $x$ and $y$ must be $0$, which implies that the only point outside of $\conj{V}\cap U_w$ that $\tilde{f}$ and $\tilde{g}$ vanish at is $[0:0:0:1]$. 

Take any $H\in\tilde{I}$. We now wts that $H$ vanishes at $[0:0:0:1]$. We know that $H=\tilde{h}w^n$, where $h$ is generated by $f$ and $g$. Since the $z^2$ term in $g$ has a lower degree than $y^3$, there will always be a homogenizing $w$ term in front of any monomial not containing any $x$s or $y$s in $\tilde{h}$, which means that $H$ necessarily vanishes.
\subsection*{3.3.3}
Suppose we have some homogeneous polynomial $f$ such that $f^n\in\tilde{I}$. Then, if we evaluate at $x_0=1$, we have $\conj{f}^n\in I$, implying that $\conj{f}\in I$. Applying the homogenization process to $\conj{f}$, we get that $f\in\tilde{I}$. For a general polynomial $g$ such that $g^n\in\tilde{I}$, write $g=g^{(0)}+g^{(1)}+\cdots+g^{(d)}$. We have that the maximum degree of a monomial in $g^n=h$ is $nd$, so we can write $h=h^{(0)}+\cdots+h^{(nd)}$. Since $h\in\tilde{I}$, all the homogeneous components of $h$ are also in $\tilde{I}$. In particular, $h^{(0)}={g^{(0)}}^n\in\tilde{I}$, which means $g^{(0)}\in\tilde{I}$. Further, $h^{(n)}$ can be written as ${g^{(1)}}^n$ plus more terms which all have $g^{(0)}$ in them, which implies that $g^{(1)}\in\tilde{I}$. Proceeding this way inductively, we will have that $g\in\tilde{I}$.
%Since the homogenization of an ideal $\~{I}$ is homogeneous, take a set of homogeneous polynomials $\{h_i\}$ as its generators. Suppose that $f^n\in\~{I}$ for some $n$. Then $f^n=\sum a_ih_i$ for some polynomials $a_i$
\subsection*{3.4.3}
Consider the curves defined by $V=Z(xz-y^2)\subset\proj^2$ and $W=Z(x)$. The latter is a copy of the projective line in $\proj^2$. As in the example in the book, define $\phi:W\to V, [0:y:z]\mapsto [y^2:yz:z^2]$ and $\phi^{-1}:[x:y:z]\mapsto
\begin{cases}
    [0:x:y],&x\neq0\\
    [0:y:z],&z\neq0
\end{cases}$. 
We have $\phi\circ\phi^{-1}([x:y:z])=\phi([0:x:y])=[x^2:xy:y^2]=[x^2:xy:xz]=[x:y:z]$ if $x\neq0$ and $=\phi([0:y:z])=[y^2:yz:z^2]=[xz:yz:z^2]=[x:y:z]$ if $z\neq0$. The other way, we have $\phi^{-1}\circ\phi([0:y:z])=\phi^{-1}([y^2:yz:z^2])=[0:y^2:yz]=[0:y:z]$ if $y\neq0$ and $=[0:yz:z^2]=[0:y:z]$ if $z\neq0$. However, these are not projectively equivalent, since their coordinate rings are $\cn[x,y,z]/(xz-y^2)$ and $\cn[x,y,z]/(x)\equiv\cn[y,z]$ are not isomorphic.
\subsection*{3.4.4}
We can show that the affine varieties are isomorphic by exhibiting an isomorphism: Let $\phi(x,y,z)=(y^4+z^4,y,z)$ and $\phi^{-1}(x,y,z)=(x-y^4-z^4,y,z)$. For $(0,y,z)$ on $V(x)$, we have $\phi^{-1}\circ\phi(0,y,z)=\phi^{-1}(y^4+z^4,y,z)=(0,y,z)$, and for $(x,y,z)$ on $V(x-y^4-z^4)$, $\phi\circ\phi^{-1}(x,y,z)=\phi(x-y^4-z^4,y,z)=\phi(0,y,z)=(y^4+z^4,y,z)=(x,y,z)$.
%between their coordinate rings. Note that the coordinate ring of $V(x)$ is isomorphic to $\cn[y,z]$. Define a map $\cn[x,y,z]\to\cn[y,z]$ as $x\mapsto y^4+z^4, y\mapsto y, z\mapsto z$. This is surjective as seen by restricting to the set of polynomials which do not involve x. The kernel of this map is equal to $(x-y^4-z^4)$ since $x-y^4-z^4$ is irreducible, so we have that the coordinate rings are isomorphic.

The projective closure of $V(x)$ is the zero set of the homogenization of the ideal $(x)$ in $\cn[w,x,y,z]$, which is still $(x)$ since $x$ is homogeneous and $(x)$ is maximal. Thus the closure of $V(x)$ is isomorphic to $\proj^2$, which has no singular points. As for the projective closure of $V(x-y^4-z^4)$, since the ideal is principal, we can just take the homogenization of the single generator to get the homogeneous ideal, $(w^3x-y^4-z^4)$. Looking at the affine cone in $\cn^4$, we have a dimension 3 affine variety (since the affine cone is a hypersurface).If we examine the tangent space at the point $(0,1,0,0)$, we have that the dimension of the tangent space is $4-\rk([2w^2x,w^3-y^4-z^4,-3y^3,-3z^3])=4$, since the Jacobian at $(0,1,0,0)$ is a zero matrix. Thus, $(0,1,0,0)$ is a singular point, which corresponds to a singular point on the projective variety from whence the affine cone came, so the two projective closures are not isomorphic.
\subsection*{3.5.1}
A general homogeneous quadratic polynomial is of the form $f(x,y,z)=\openm x&y&z\closem A\openm x\\y\\z\closem$, where $A$ is a symmetric matrix. It is reducible only if it is the product of two linear factors, $(b_1x+b_2y+b_3z)(c_1x+c_2y+c_3z)=b_1c_1x^2+b_2c_2y^2+b_3c_3z^2+(b_1c_2+b_2c_1)xy+(b_1c_3+b_3c_1)xz+(b_2c_3+b_3c_2)yz$, which corresponds to $A$ being singular.

If we attempt a linear change of coordinates $C$ to the zero set of $f$, we get the zero set of a polynomial $(Cx)^TA(Cx)=x^TC^TACx$, where $x$ represents the column vector containing the coordinates here. Thus, our problem is now to find, for each nonsingular, symmetric $A$, $B$, a nonsingular matrix $C$ such that $B=C^TAC$. We can just as easily set $A$ to be the 3 by 3 identity, as we can then just compose and invert. Therefore, for each nonsingular symmetric $B$, we want to find a nonsingular $C$ such that $B=C^TC$. From Gauss's decomposition, we know that any quadratic form is equivalent to a diagonal form, so we can just consider the case of diagonal $B$. Then, since we can multiply by a constant scaling factor, set the first entry of $B$ to be $1$ and the other two to be $a$ and $b$. Note that neither $a$ nor $b$ can be zero, as that would make $B$ singular. Take $C$ to be diagonal, with entries $1, a^{-1/2}$, and $b^{-1/2}$. 
\subsection*{3.5.2}
For any irreducible quadratic $f$ in $\cn[x,y]$, consider its homogenization. The homogenized polynomial must also be irreducible, for otherwise evaluating at $1$ would result in a factoring of the original polynomial. Then, we have a linear change of coordinates that makes the projective closure of $V(f)$ isomorphic to $Z(xz-y^2)$. The linear change of coordinates defines an isomorphism between $V(f)$, which is one of the affine patches of $\conj{V(f)}$, and one of the affine patches of $Z(xz-y^2)$. Since the affine patches of $Z(xz-y^2)$ is either a parabola or a hyperbola, we have that any irreducible conic is isomorphic to one of these.

Consider the coordinate ring of the parabola, $\cn[x,y]/(x-y^2)$. The book shows that the parabola is isomorphic to the affine line in section 1.3, so we know that this is isomorphic to $\cn[x]$. The coordinate ring of the hyperbola is $\cn[x,y]/(xy-1)$. Consider the map $\cn[x,y]\to\cn[x,1/x]$ where $x\mapsto x$ and $y\mapsto 1/x$. This map is surjective since we're mapping generators to generators. To show that it's injective, we note that since the hyperbola is a dimension 1 irreducible variety, all ideals containing $(xy-1)$ must be of the form $(x-a,y-b)$, but the map can't vanish on elements of the form $x-a$. Thus, the kernel of the map is $(xy-1)$, and we have an isomorphism. Now, we can show that the coordinate rings are not isomorphic, as any homomorphism from $\cn[x,1/x]$ must take $x$ to a unit.
\subsection*{3}
($\implies$): Suppose $I$ can be generated by homogeneous polynomials. Then for any polynomial $f$ we can write $f=\sum a_ih_i$, where $h_i$ are the homogeneous generators. If we let $d_i$ be the degree of $h_i$, we have that $f^{(d)}=\sum a_i^{(d-d_i)}h_i$ (if $d-d_i<0$, don't include that term). Each of these terms has degree $d$, and summing them up gives $f$. Thus, we get that $f^{(d)}$ can be generated by the homogeneous generators.

($\impliedby$): For some set of generators of $I$, take the set of homogeneous components of all the generators. Then these still generate the ideal and are all homogeneous.
\subsection*{4}
a. Lines in $\proj^2$ correspond to subspaces of $\aff^3$ of dimension 2. Thus, if three points do not lie on a line in $\proj^2$, then their representatives (take any of them) in $\aff^3$ form a basis for $\aff^3$. If we want a linear change of coordinates, we can just take the change-of-basis matrix between the bases determined by the $x_i$ and the $y_i$. However, if we have 4 points, their representatives are guaranteed to be linearly dependent, which means that the result of a linear map on the fourth point will be determined by its result on the other three, so we can't pick an arbitrary 4 points to map to.

b. Look at the subspaces corresponding to these lines. Since a line is determined by two distinct points, we can look at the subspace generated by all points in the line. Using this, three lines intersecting at the same point means that the three subspaces contain a common nonzero element. Now, since a 2-dimensional subspace is determined by its orthogonal complement, we want to show that if three subspaces don't contain a common nonzero element, then vectors from their orthogonal complements will be linearly independent, for then we can use the argument from part (a) to show that there exists a change of basis. 

Suppose we have a linear dependence in the orthogonal complements. To be more specific, let $v_1=av_2+bv_3$. For any point $c$ in the orthogonal complement of $v_1$, we have $\langle v_1,c\rangle=0=a\langle v_2,c\rangle+b\langle v_3,c\rangle$. Since inner-producting with $v_2$ is a linear operator from $\aff^2$ to $\aff^1$, there is some nonzero $c$ such that $\langle v_2,c\rangle=0$, which would them imply that $\langle v_3,c\rangle=0$, which means that $c$ is in all 3 of the 2-dimensional subspaces.
\end{document}
