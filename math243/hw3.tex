\documentclass{article}
\usepackage{geometry}
\usepackage[namelimits,sumlimits]{amsmath}
\usepackage{amssymb,amsfonts}
\usepackage{multicol}
\usepackage{mathrsfs}
\usepackage[cm]{fullpage}
\newcommand{\nc}{\newcommand}
\newcommand{\tab}{\hspace*{5em}}
\newcommand{\conj}{\overline}
\newcommand{\dd}{\partial}
\nc{\cn}{\mathbb{C}}
\nc{\rn}{\mathbb{R}}
\nc{\qn}{\mathbb{Q}}
\nc{\zn}{\mathbb{Z}}
\nc{\aff}{\mathbb{A}}
\nc{\proj}{\mathbb{P}}
\nc{\pd}[2]{\frac{\partial {#1}}{\partial {#2}}}
\nc{\ep}{\epsilon}
\nc{\topo}{\mathscr{T}}
\nc{\basis}{\mathscr{B}}
\nc{\nullset}{\varnothing}
<<<<<<< HEAD
\nc{\conj}{\overline}
=======
>>>>>>> 65bdcfe6800144d21eca8231f8c60def9649a6ad
\setlength{\parindent}{0mm}
\begin{document}
Name: Hall Liu

Date: \today 
\vspace{1.5cm}

<<<<<<< HEAD
\subsection*{3.2.1}
Let $\{U_i\}$ be an open cover of a projective variety $Z$. Consider the preimage of $Z$ in $\aff^{n+1}$ and each of the open covering sets under the quotient map. Consider the intersection of $\pi^{-1}(Z)$ with the set of points with absolute value $1$. This being a closed and bounded set, it is compact, so there exists a finite subcover by preimages of the $U_i$. Then, the finite set of the $U_i$ from this subcover also cover $Z$, since $[z_0:\ldots z_n]\in Z\implies 1/|z|(z_0,\ldots,z_n)\in\pi^{-1}(U_i)$ for some $i$, and taking the image under $\pi$ gives that $[z_0:\ldots:z_n]\in U_i$.

We can find an affine variety embedded inside a projective variety $Z$ by considering the points of $Z$ where $x_0\neq0$. If $Z$ is defined by a set of homogenous polynomials $\{F_i\}$, then the affine variety $V$ is defined by the set of polynomials obtained by evaluating each of the $F_i$ at $x_0=1$. We already have that $Z$ is compact in both the Euclidean and Zariski toplogies (Zariski is coarser than Euclidean), so we need to show that $\bar{V}=Z$ in the Euclidean topology. Consider $Z-V$. This is the set of points where $x_0=0$. Looking at the preimage in $\aff^{n+1}$, the preimage of $Z-V$ is the preimage of $Z$ intersected with a hyperplane, so all such points are limit points of the preimage of $V$, which makes the closure of $V$ $Z$.
\subsection*{3.2.2}
Call the three variables $x,y,$ and $z$. To go from a homogenous polynomial of degree $d$ in 3 variables to a polynomial in 2 variables, evaluate at $z=1$ ($\psi$). To go the other way, multiply each monomial term by a power of $z$ so that it has degree $d$ ($\psi^{-1}$). This is a bijection, as $\psi^{-1}\circ\psi$ is just adding on some number of $z$ terms then taking them away again by making them $1$, so it's the identity. $\psi$ on a monomial $x^ay^bz^c$ ($a+b+c=d$) takes it to $x^ay^b$, then $\psi^{-1}$ takes that to $x^ay^bz^c$ again. 

In the affine patches, the subspace topology has closed sets consisting of $V\cap W'\cap\aff^2$, where $V'$ is some other projective variety. The affine Zariski topology has closed sets consisting of $V\cap\aff^2\cap W$, where $W$ is some affine variety in $\aff^2$. For each affine variety $W$, we can produce a projective variety $W'$ by taking $V(\psi^{-1}(I(W)))$. $W'$ has the same zero set as $W$ when restricted to the affine patch of $\proj^2$, since $f(w_1,w_2,1)=0$ iff $\psi(f)(w_1,w_2)=0$. Thus we have $V\cap\aff^2\cap W=V\cap W'\cap\aff^2$. Similarly, for each projective variety $W'$, we can produce an affine variety $W$ in the same manner, and we have that each closed set of the form $V\cap W'\cap\aff^2$ is equal to one of the form $V\cap\aff^2\cap W$, showing that the topologies are equivalent.
\subsection*{3.3.1}
Suppose we can write $\overline{V}=Y\cup Z$, with both $Y,Z$ being proper subvarieties of $\conj{V}$. Then, we have $V=\conj{V}\cap U_0=(Y\cap U_0)\cup(Z\cap U_0)$. Since $\conj{V}$ is the smallest closed set containing $V$, both $Y-V$ and $Z-V$ must be nonempty, so $Y\cap U_0$ and $Z\cap U_0$ are proper affine subvarieties of $V$, which contradicts $V$'s irreducibility.
\subsection*{3.3.2}
Plugging the parametrization $(t,t^2,t^3)$ into these polynomials yields zero in both cases. Going the other way, $f=y-x^2$ requires that any solution have the form $(x,x^2,z)$, and $g=z^2-2xyz+y^3=0$ requires that $z^2-2x^3z+x^6=0\implies(z-x^3)^2=0\implies z=x^3$, so we have that all solutions are of the form $(x,x^2,x^3)$. When we homogenize, we get the polynomials $\~{f}=wy-x^2$ and $\~{g}=wz^2-2xyz+y^3$. Since both of these are in $\~{I}$, we have that $\conj{V}\subset V(\~{f},\~{g})$. To show the converse, we want to show that $\~{f}$ and $\~{g}$ only vanish at points $[0:x:y:z]$ in the set of limit points of $V$ in $\proj^3$. If we set $w=0$, we get that both $x$ and $y$ must be $0$, which implies that the only point outside of $\conj{V}\cap U_w$ that $\~{f}$ and $\~{g}$ vanish at is $[0:0:0:1]$. 

Take any $H\in\~{I}$. We now wts that $H$ vanishes at $[0:0:0:1]$. We know that $H=\~{h}w^n$, where $h$ is generated by $f$ and $g$. Since the $z^2$ term in $g$ has a lower degree than $y^3$, there will always be a homogenizing $w$ term in front of any monomial not containing any $x$s or $y$s in $\~{h}$, which means that $H$ necessarily vanishes.
\subsection*{3.3.3}
Suppose we have some homogeneous polynomial $f$ such that $f^n\in\~{I}$. Then, if we evaluate at $x_0=1$, we have $\conj{f}^n\in I$, implying that $\conj{f}\in I$. Applying the homogenization process to $\conj{f}$, we get that $f\in\~{I}$. For a general polynomial $g$ such that $g^n\in\~{I}$, write $g=g^{(0)}+g^{(1)}+\cdots+g^{(d)}$. We have that the maximum degree of a monomial in $g^n=h$ is $nd$, so we can write $h=h^{(0)}+\cdots+h^{(nd)}$. Since $h\in\~{I}$, all the homogeneous components of $h$ are also in $\~{I}$. In particular, $h^{(0)}={g^{(0)}}^n\in\~{I}$, which means $g^{(0)}\in\~{I}$. Further, $h^{(n)}$ can be written as ${g^{(1)}}^n$ plus more terms which all have $g^{(0)}$ in them, which implies that $g^{(1)}\in\~{I}$. Proceeding this way inductively, we will have that $g\in\~{I}$.
%Since the homogenization of an ideal $\~{I}$ is homogeneous, take a set of homogeneous polynomials $\{h_i\}$ as its generators. Suppose that $f^n\in\~{I}$ for some $n$. Then $f^n=\sum a_ih_i$ for some polynomials $a_i$
\subsection*{3.4.3}
Consider the curves defined by $V=Z(xz-y^2)\subset\proj^2$ and $W=Z(x)$. The latter is a copy of the projective line in $\proj^2$. As in the example in the book, define $\phi:W\to V, [0:y:z]\mapsto [y^2:yz:z^2]$ and $\phi^{-1}:[x:y:z]\mapsto\begin{cases}[0:x:y],&x\neq0\\[0:y:z],&z\neq0\end{cases}$. We have $\phi\circ\phi^{-1}([x:y:z])=\phi([0:x:y])=[x^2:xy:y^2]=[x^2:xy:xz]=[x:y:z]$ if $x\neq0$ and $=\phi([0:y:z])=[y^2:yz:z^2]=[xz:yz:z^2]=[x:y:z]$ if $z\neq0$. The other way, we have $\phi^{-1}\circ\phi([0:y:z])=\phi^{-1}([y^2:yz:z^2])=[0:y^2:yz]=[0:y:z]$ if $y\neq0$ and $=[0:yz:z^2]=[0:y:z]$ if $z\neq0$. However, these are not projectively equivalent, since their coordinate rings are $\cn[x,y,z]/(xz-y^2)$ and $\cn[x,y,z]/(x)\equiv\cn[y,z]$ are not isomorphic.
\subsection*{3.4.4}
We can show that the affine varieties are isomorphic by exhibiting an isomorphism between their coordinate rings. Note that the coordinate ring of $V(x)$ is isomorphic to $\cn[y,z]$. Define a map $\cn[x,y,z]\to\cn[y,z]$ as $x\mapstoy^4+z^4, y\mapsto y, z\mapsto z$. This is surjective as seen by restricting to the set of polynomials which do not involve x. The kernel of this map is equal to $(x-y^4-z^4)$ since $x-y^4-z^4$ is irreducible, so we have that the coordinate rings are isomorphic.

The projective closure of $V(x)$ is the zero set of the homogenization of the ideal $(x)$ in $\cn[w,x,y,z]$, which is still $(x)$ since $x$ is homogeneous and $(x)$ is maximal. Thus the closure of $V(x)$ is isomorphic to $\proj^2$, which has no singular points. 
\end{document}
