\documentclass{article}
\usepackage{geometry}
\usepackage[namelimits,sumlimits]{amsmath}
\usepackage{amssymb,amsfonts}
\usepackage{multicol}
\usepackage{mathrsfs}
\usepackage[cm]{fullpage}
\newcommand{\nc}{\newcommand}
\newcommand{\tab}{\hspace*{5em}}
\newcommand{\conj}{\overline}
\newcommand{\dd}{\partial}
\nc{\cn}{\mathbb{C}}
\nc{\rn}{\mathbb{R}}
\nc{\qn}{\mathbb{Q}}
\nc{\zn}{\mathbb{Z}}
\nc{\aff}{\mathbb{A}}
\nc{\proj}{\mathbb{P}}
\nc{\pd}[2]{\frac{\partial {#1}}{\partial {#2}}}
\nc{\ep}{\epsilon}
\nc{\topo}{\mathscr{T}}
\nc{\basis}{\mathscr{B}}
\nc{\nullset}{\varnothing}
\nc{\openm}{\begin{pmatrix}}
\nc{\closem}{\end{pmatrix}}
\nc{\conj}{\overline}
\setlength{\parindent}{0mm}
\begin{document}
Name: Hall Liu

Date: \today 
\vspace{1.5cm}
\subsection*{4.2.1}
Using the notation of the discussion in the beginning of (4.2), if $f$ vanishes nowhere on $V$, then we have $U=V$ and therefore $\cn[V][1/f]\equiv\cn[W]\equiv\cn[V]$. This implies that $1/f$ is an element of $\cn[V]$, so $f$ is invertible.
\subsection*{4.2.2}
Let $V$ be the whole space, $\aff^2$. Its coordinate ring is $\cn[x,y]$. Take $f=x$ and $g=y$ in the coordinate ring. Then we have that $\aff^2-V(x)$ and $\aff^2-V(y)$ are both open affine sets by the discussion in the beginning of (4.2). These two sets correspond to the affine plane with the $y$- and $x$-axis removed, resp., so together they cover $\aff^2-\{0\}$.
\subsection*{4.2.3}
We can identify each point in $\aff^{n^2}$ with a $n\times n$ matrix and vice versa. The general linear group corresponds to those matrices which have nonzero determinant, or the complement in $\aff^{n^2}$ of the hypersurface defined by $V(\Delta)$, where $\Delta$ is the determinant. Thus, the general linear group is an open affine set.
\subsection*{4.3.1}
First, change coordinates to make the lines and point nicer. First apply a translation to move $p$ to $(0,0)$, then apply a rotation to bring $l'$ in line with the $y$-axis, then apply a scaling to bring $l$ to the line $x=1$. These three actions are all actions of $\text{Aff}(\aff^2)$ on $\aff^2$, so their composition is an automorphism of $\aff^2$. Call this automorphism $\phi$. Denote the projection map by $\psi$. For any $q\not\in l'$, $q'=phi(q)$ has $x$-coordinate not equal to $0$. Then, $\psi(q')=y/x$, where $q'=(x,y)$ and we identify points on $l$ with their $y$-coordinate and $(0,0),q',\text{ and }(1,y/x)$ lie on a line. Now we can translate back by applying $\phi^{-1}$, and we have that $\psi(q)=y/x$ still if we identify $l$ with $\aff^1$ by taking the old identification under $\phi^{-1}$, since affine transformations preserve lines. Since $\phi$ was a polynomial map, $y$ and $x$ are polynomial functions of the coordinates of $q$ with $x$ nonzero, so we have that the map is regular.
\subsection*{4.3.2}
The punctured plane is covered by the two open sets $U=\aff^2-V(x)$ and $V=\aff^2-V(y)$. If we let $g$ be a regular function on the punctured plane, then $g|_U=\frac{h_1}{k_1}$ and $g|_V=\frac{h_2}{k_2}$ for $h_i,k_i\in\cn[x,y]$. Now, $k_1$ and $k_2$ can't vanish simultaneously anywhere but $(0,0)$, so the ideal $(k_1,k_2)$ must contain $(x,y)$. Thus, we can write $xg=(l_1k_1+l_2k_2)\frac{h_1}{k_1}$, and by the technique used in the proof in the book, we have that $g$ agrees with a function of the form $\frac{f_1}{x}$ on $U$. Similarly, $g$ also agrees with a function of the form $\frac{f_2}{y}$ on $V$, where $f_1,f_2\in\cn[x,y]$. This means that $\frac{f_1}{x}=\frac{f_2}{y}$ on the dense set $\aff^2-V(x,y)$, so $f_1y=f_2x$ as polynomials. However, since $x,y$ are relatively prime, $f_1$ must be divisible by $x$ and $f_2$ must be divisible by $y$, so $g$ turns out to be a polynomial when restricted to $U$ and $V$. This shows that the ring of regular functions on $\aff^2-\{0\}$ is contained in $\cn[x,y]$. Conversely, each function in $\cn[x,y]$ is a regular function, so we have that the two rings are equal. 

Now, if $\aff^2-\{0\}$ were an affine variety, it'd be isomorphic to $\aff^2$ since they have the same coordinate rings. The isomorphism would induce an automorphism of $\cn[x,y]$, which take generators to generators. Let this automorphism take $x$ to $f_1$ and $y$ to $f_2$ ($f_1$ and $f_2$ are not nec. the same as above), where $f_1$ and $f_2$ generate the ideal $(x,y)$. Then, the isomorphism from $\aff^2-\{0\}$ to $\aff^2$ induced by this automorphism looks like $(f_1(x,y),f_2(x,y)$. However, $f_1$ and $f_2$ can't simultaneously vanish on $\aff^2-\{0\}$ because they're generators, which means that nothing can get mapped to $0$, which is a contradiction.

\end{document}
