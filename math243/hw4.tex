\documentclass{article}
\usepackage{geometry}
\usepackage[namelimits,sumlimits]{amsmath}
\usepackage{amssymb,amsfonts}
\usepackage{multicol}
\usepackage{mathrsfs}
\usepackage[cm]{fullpage}
\newcommand{\nc}{\newcommand}
\newcommand{\tab}{\hspace*{5em}}
\newcommand{\conj}{\overline}
\newcommand{\dd}{\partial}
\nc{\cn}{\mathbb{C}}
\nc{\rn}{\mathbb{R}}
\nc{\qn}{\mathbb{Q}}
\nc{\zn}{\mathbb{Z}}
\nc{\aff}{\mathbb{A}}
\nc{\proj}{\mathbb{P}}
\nc{\pd}[2]{\frac{\partial {#1}}{\partial {#2}}}
\nc{\ep}{\epsilon}
\nc{\topo}{\mathscr{T}}
\nc{\basis}{\mathscr{B}}
\nc{\nullset}{\varnothing}
\nc{\openm}{\begin{pmatrix}}
\nc{\closem}{\end{pmatrix}}
\begin{document}
Name: Hall Liu

Date: \today 
\vspace{1.5cm}
\subsection*{4.2.1}
Using the notation of the discussion in the beginning of (4.2), if $f$ vanishes nowhere on $V$, then we have $U=V$ and therefore $\cn[V][1/f]\equiv\cn[W]\equiv\cn[V]$. This implies that $1/f$ is an element of $\cn[V]$, so $f$ is invertible.
\subsection*{4.2.2}
Let $V$ be the whole space, $\aff^2$. Its coordinate ring is $\cn[x,y]$. Take $f=x$ and $g=y$ in the coordinate ring. Then we have that $\aff^2-V(x)$ and $\aff^2-V(y)$ are both open affine sets by the discussion in the beginning of (4.2). These two sets correspond to the affine plane with the $y$- and $x$-axis removed, resp., so together they cover $\aff^2-\{0\}$.
\subsection*{4.2.3}
We can identify each point in $\aff^{n^2}$ with a $n\times n$ matrix and vice versa. The general linear group corresponds to those matrices which have nonzero determinant, or the complement in $\aff^{n^2}$ of the hypersurface defined by $V(\Delta)$, where $\Delta$ is the determinant. Thus, the general linear group is an open affine set.
\subsection*{4.3.1}
First, change coordinates to make the lines and point nicer. First apply a translation to move $p$ to $(0,0)$, then apply a rotation to bring $l'$ in line with the $y$-axis, then apply a scaling to bring $l$ to the line $x=1$. These three actions are all actions of $\text{Aff}(\aff^2)$ on $\aff^2$, so their composition is an automorphism of $\aff^2$. Call this automorphism $\phi$. Denote the projection map by $\psi$. For any $q\not\in l'$, $q'=phi(q)$ has $x$-coordinate not equal to $0$. Then, $\psi(q')=y/x$, where $q'=(x,y)$ and we identify points on $l$ with their $y$-coordinate and $(0,0),q',\text{ and }(1,y/x)$ lie on a line. Now we can translate back by applying $\phi^{-1}$, and we have that $\psi(q)=y/x$ still if we identify $l$ with $\aff^1$ by taking the old identification under $\phi^{-1}$, since affine transformations preserve lines. Since $\phi$ was a polynomial map, $y$ and $x$ are polynomial functions of the coordinates of $q$ with $x$ nonzero, so we have that the map is regular.
\subsection*{4.3.2}
The punctured plane is covered by the two open sets $U=\aff^2-V(x)$ and $V=\aff^2-V(y)$. If we let $g$ be a regular function on the punctured plane, then $g|_U=\frac{h_1}{k_1}$ and $g|_V=\frac{h_2}{k_2}$ for $h_i,k_i\in\cn[x,y]$. Now, $k_1$ and $k_2$ can't vanish simultaneously anywhere but $(0,0)$, so the ideal $(k_1,k_2)$ must contain $(x,y)$. Thus, we can write $xg=(l_1k_1+l_2k_2)\frac{h_1}{k_1}$, and by the technique used in the proof in the book, we have that $g$ agrees with a function of the form $\frac{f_1}{x}$ on $U$. Similarly, $g$ also agrees with a function of the form $\frac{f_2}{y}$ on $V$, where $f_1,f_2\in\cn[x,y]$. This means that $\frac{f_1}{x}=\frac{f_2}{y}$ on the dense set $\aff^2-V(x,y)$, so $f_1y=f_2x$ as polynomials. However, since $x,y$ are relatively prime, $f_1$ must be divisible by $x$ and $f_2$ must be divisible by $y$, so $g$ turns out to be a polynomial when restricted to $U$ and $V$. This shows that the ring of regular functions on $\aff^2-\{0\}$ is contained in $\cn[x,y]$. Conversely, each function in $\cn[x,y]$ is a regular function, so we have that the two rings are equal. 

Now, if $\aff^2-\{0\}$ were an affine variety, it'd be isomorphic to $\aff^2$ since they have the same coordinate rings. The isomorphism would induce an automorphism of $\cn[x,y]$, which take generators to generators. Let this automorphism take $x$ to $f_1$ and $y$ to $f_2$ ($f_1$ and $f_2$ are not nec. the same as above), where $f_1$ and $f_2$ generate the ideal $(x,y)$. Then, the isomorphism from $\aff^2-\{0\}$ to $\aff^2$ induced by this automorphism looks like $(f_1(x,y),f_2(x,y))$. However, $f_1$ and $f_2$ can't simultaneously vanish on $\aff^2-\{0\}$ because they're generators, which means that nothing can get mapped to $0$, which is a contradiction.
\subsection*{3}
Since every affine automorphism is given by $Ax+b$, where $A$ is a $n\times n$ matrix, an affine automorphism is determined by where it sends the basis vectors $e_1,\ldots,e_n$ and by where it sends $0$. Thus, we want to find an equivalence class of $(n+1)\times(n+1)$ matrices which send the corresponding points in the affine patch to the right places. If we fix the affine patch as the set of points where $x_0\neq0$, the point in projective space corresponding to $0\in\aff^n$ is $[1:0:\cdots:0]$, and this should get mapped to $[1:b_1:\cdots:b_n]$. Similarly, the points corresponding to the standard basis vectors look like $[1:0:\cdots:1:0:\cdots:0]$ (with a $1$ in the $i$th position), and these should get mapped to $[1:a_{i1}+b_1:a_{i2}+b_2:\cdots:a_{in}+b_n]$. 

Now, we want to construct a matrix over $\aff^{n+1}$ that does these mappings. The following matrix does the job:
$$
\begin{pmatrix}
    1&0&0&\hdots&0\\
  b_1&a_{11}&a_{12}&\hdots&a_{1n}\\
  b_2&a_{21}&a_{22}&\hdots&a_{2n}\\
    \vdots&\vdots&\vdots&\ddots&\vdots\\
  b_n&a_{n1}&a_{n2}&\hdots&a_{nn}\\
\end{pmatrix}
$$

Note that this is actually a representative of the equivalence class, as we've already fixed the projective scaling by setting the first coordinate of the image to $1$. This is a nonsingular matrix, since we can expand by minors along the first row and get that the determinant of this is the same as $\det A$. 
\subsection*{4}
Since the way $F$ is defined immediately makes it a polynomial map, we just need to find a polynomial inverse to show that it's an automorphism. Let the inverse be $F^{-1}(s,t)=(s-f(t),t)$. Composing $F\circ F^{-1}(s,t)$ gives $F(s-f(t),t)=(s,t)$ and similarly the other way around. Affine transformations are of the form $A\openm s\\t\closem+\openm b_1\\b_2\closem$, so if $F$ were affine, it'd need to be of the form $(a_{11}s+a_{12}t+b_1,a_{21}s+a_{22}t+b_2)$, so we'd actually need $f$ to be degree $1$ in order for $F$ to be affine. Thus, since there are degree $>1$ polynomials in $k[x]$, $\text{Aut}(\aff^2)$ is larger than the affine group.
\subsection*{5}
a. Since the ideal of $X$ is principally generated, its homogenization must be generated by the homogenization of the principal generator, or $zy^2-(x^3+axz^2+bz^3)$. The Zariski closure of $X$ is thus the zero set of the aforementioned homogenized polynomial.

\noindent b. If we look at the zero set when we let $x\neq0$ in the polynomial above (set $x=1$), we get $zy^2-(1+az^2+bz^3)=0$ or $zy^2=bz^3+az^2+1$. 

\noindent c. If we look at the affine patch defined by $y\neq0$, we have that the morphism $\tilde{G}$ restricted to this affine patch becomes $[x:1:z]\mapsto[x:1]$. Since all values of $(x,z)$ are acceptable, this effectively maps the affine patch to the affine line in $\proj^1$. 

Now, if we had a morphism $H$ defined on all of $\proj^2$, it would need to agree with $\tilde{G}$ where it is defined. Thus, we'd need that $H([x:0:y])=[1:0]$ for all $x\neq0$ and $H([0:y:z])=[0:1]$ for all $y\neq0$. Since $H$ is continuous and it is constant on a dense subset of $[x:0:z]$ and $[0:y:z]$ (no restrictions on $x,y$), it must be constant on the whole of that set, which means that we can't agree on a value for $[0:0:1]$.

\noindent d. First, note that we don't have a problem if $b\neq0$, since then we'd have that $[0:0:1]$ is not in $\bar{X}$, so we can just use $\tilde{G}$ from above as our morphism. Thus, for the rest of this part, assume $b=0$. On the affine slice containing $X$ (and the slice $x\neq0$), we can define $G$ as $[x:y:z]\mapsto[x:z]$, since $z$ is nonzero there. For anything in $\conj{X}-X$, we must have $x=0$, as setting $z=0$ in $zy^2-(x^3+axz^2)$ gives $x^3$. Thus, $\conj{X}-X$ is the single point $[0:1:0]$, contained in the affine slice $y\neq0$. 

On the slice $y\neq0$, let $G$ map $[x:y:z]$ to $[y^2-axz:x^2]$. This is well defined when $y\neq0$, as $x=0$ implies that the first coordinate is nonzero and $x\neq0$ implies that the second coordinate is. It agrees with $[x:z]$ on the intersection with the slice $z\neq0$, since $[x:z]=[x^3:zx^2]=[zy^2-axz^2:zx^2]=[y^2-axz:x^2]$ (we can do this because $x\neq0$ on this slice in the variety). Similarly, we can do the same on the intersection with the slice $x\neq0$, as $z\neq0$ on that slice of the variety.
\subsection*{Midterm question 4}
On the affine patches $x\neq0$ and $z\neq0$, the map is polynomial and well-defined, so no problems there. On the affine patch $y\neq0$, let the morphism be defined by $[x:y:z]\mapsto[y^2:(x-z)(x-2z)]$. This is clearly well-defined since the first coordinate is nonzero. We now need to show that it agrees with $[x:y:z]\mapsto[x:z]$ on the intersection with the other two patches. 

First, suppose $z,y\neq0$. Then if $x=0$ for a point on the variety, we'd have $zy^2=0$, which is a contradiction, so $x$ must be nonzero also on this intersection. Now, we have $[x:z]=[xy^2:zy^2]=[xy^2:x(x-z)(x-2z)]=[y^2:x(x-z)(x-2z)]$. so they agree on the intersection of these two patches. Similarly, if we suppose $x,y\neq0$, the same computation goes through. Thus, we have that $[x:y:z]\mapsto[x:z]$ is a morphism on most of $V(zy^2-x(x-z)(x-2z))$, with the point $[0:1:0]$ mapping to $[1:0]$.
\end{document}
