\documentclass{article}
\usepackage{geometry}
\usepackage[namelimits,sumlimits]{amsmath}
\usepackage{amssymb,amsfonts}
\usepackage{multicol}
\usepackage{mathrsfs}
\usepackage[cm]{fullpage}
\newcommand{\nc}{\newcommand}
\newcommand{\tab}{\hspace*{5em}}
\newcommand{\conj}{\overline}
\newcommand{\dd}{\partial}
\nc{\cn}{\mathbb{C}}
\nc{\rn}{\mathbb{R}}
\nc{\qn}{\mathbb{Q}}
\nc{\zn}{\mathbb{Z}}
\nc{\aff}{\mathbb{A}}
\nc{\pd}[2]{\frac{\partial {#1}}{\partial {#2}}}
\nc{\ep}{\epsilon}
\nc{\topo}{\mathscr{T}}
\nc{\basis}{\mathscr{B}}
\nc{\nullset}{\varnothing}
\nc{\cpr}{\cn[x_1,\ldots,x_n]}
\setlength{\parindent}{0mm}
\begin{document}
Name: Hall Liu

Date: \today 
\vspace{1.5cm}
\subsection*{1.3.1}
Since continuous functions on a space are continuous on any subspace, we just need to show that $F$ is continuous from $\aff^n$ to $\aff^m$. Since polynomial maps from $\cn^n$ to $\cn^m$ are continuous, and the Zariski topology is coarser than the Euclidean topology, we get that $F$ is continuous with hte Zariski topology.
\subsection*{1.3.2}
Consider the projection from $\aff^3$ to $\aff$ along the first coordinate, restricted to the twisted cubic. This is a polynomial map, and its inverse is $t\mapsto(t,t^2,t^3)$, as shown in 1.2.3. 
\subsection*{1.4.1}
We have that isomorphisms of varieties are homeomorphisms of the Zariski topology, since they're continuous bijections with a continuous inverse. Then, given varieties $V,W$, a longest chain $V\supset V_{d-1}\supset\cdots\supset V_0$, and an isomorphism $F$, we get a chain of the same length in $W$ by taking the image under $F$ of all the subvarieties. The images are varieties because $F$ is a closed map and the proper inclusion is preserved because $F$ is bijective. Thus we have $\dim W\geq\dim V$. Running the argument the other way with $F^{-1}$, we have $\dim V\geq\dim W$, so $\dim V=\dim W$.
\subsection*{1.4.2}
Let $\dim Y=n$, so we have a chain $Y=Y_n\supset Y_{n-1}\supset\cdots\supset Y_0$. Taking the preimage under the surjective morphism, we get a chain of $n$ varieties in $X$ due to the continuity of morphisms. The strict inclusion is preserved, since surjectivity implies that there's always some $x$ which maps to $Y_i-Y_{i-1}$ for each $i$. Thus we have $\dim X\geq\dim Y$.
\subsection*{2.3.1}
$xy=0$ and $xz=0$ happen when either $x=0$ or both $y,z=0$. Thus the variety defined by that ideal reduces to the variety defined by $x$ and that defined by $y,z$. The ideal is radical: if $p^n\in I$, then $p^n$ evaluated at $x=0$ is the zero polynomial and $p^n$ evaluated at $y,z=0$ is the zero polynomial. Since the evaluations are homomorphisms between polynomial rings (integral domains), we must have that the same holds for $p$, which implies $p\in I$. The ideal is not prime, since $x^2yz\in I$, but $x^2\not\in I$ since evaluating at $y,z=0$ does not give zero and $yz\not\in I$ since evaluating at $x=0$ does not give $0$.
\subsection*{2.3.2}
Consider a chain of varieties $V\supsetneq V_1\supsetneq\cdots$. If we look at the ideals of each of these varieties, we have $I(V)\subsetneq I(V_1)\subsetneq\cdots$ due to the one-to-one correspondence preseving the strict inclusions. Since the polynomial ring is Noetherian, we must have $I(V_i)=I(V_{i+1})$ at some point, which implies that $V_i=V_{i+1}$, so the chain of varieties must terminate.
\subsection*{2.3.3}
For any radical ideal $I\in\cn[x_1,\ldots,x_n]$, the maximal ideal $(x_1-a_1,\ldots,x_n-a_n)$ contains $I$ iff $(a_1,\ldots,a_n)\in V(I)$. If we take the intersection $J$ of all such maximal ideals, we have that $V(I)=V(J)$. We know that $J\supset I$ by the definition of $J$, and $\sqrt{I}=I=\sqrt{J}$, so $I\supset J$, and we have $I=J$.
\subsection*{2.3.4}
Take some open cover $\{U_i\}_{i\in\mathscr{I}}$. The complements of the $U_i$ are each algebraic varieties $V_i$, and the covering nature of the open sets implies that the intersection of the varieties is empty, which, by the weak Nullstellensatz, implies that the ideal generated by all the $I(V_i)$ is the whole ring. The $I(V_i)$ are each generated by $g_{i1},\ldots,g_{in}$, so the $g_{ij}$ generate the whole ring. By the Noetherian property, we can find a finite collection of the $g_{ij}$ that generate the whole ring, which in turn corresponds to a finite collection of the $I(V_i)$ by including $I(V_i)$ if one of its generators is in the subcollection of the $g_{ij}$. We then get a finite collection of $V_i$ with empty intersection, which corresponds to a finite subcover.
\subsection*{2.3.5}
As before, every open cover $\{U_i\}_{i\in\mathscr{I}}$ of the point-complement corresponds to varieties $V_i$ which intersect either nowhere or at the single point. If they intersect nowhere, recycle the argument from above and we're done. If they intersect at a single point. Then, we still have that the $I(V_i)$ are each generated by $g_{i1},\ldots,g_{in}$, so the $g_{ij}$ together generate the ideal $(x_1-a_1,\ldots,x_n-a_n)$. By the Noetherian property, there's a finite number of the $g_{ij}$ that generate the ideal, which implies (following the argument above) that there's a finite collection of $V_i$ intersecting at that point, implying that there's a finite subcover.
\subsection*{2.3.6}
Suppose that we had a polynomial $f$ which is zero on the zero set of $y-e^x$. Evaluate this polynomial at $y=re^{i\theta}$, so the evaluated polynomial in $x$ is zero whenever $e^x=re^{i\theta}$. Since for each $y\neq0$ there are infinitely many such $x$, the evaluated polynomial must be zero, so $f(x,y)=0$ for each $y\neq 0$. By continuity, the polynomial must be zero when $y=0$ also, which makes it the zero polynomial. Thus, since the zero set of $y-e^x$ is clearly not $\aff^2$, it is not an affine variety.
\subsection*{2.4.1}
Finitely generated: Since every coordinate ring is isomorpic to $\cpr/I$, where $I$ is a radical ideal, we can generate the coordinate ring by taking the images of $x_1,\ldots,x_n$ under the quotient map.

Reduced: Suppose we had some $a\in\cpr$ such that $\bar{a}^n=0$ in $\cpr/I$. Then we have that $a^n\in I$ which implies $a\in I$ since $I$ is radical, which implies that $\bar{a}=0$. Thus there can be no nonzero nilpotents in $\cpr/I$.
\end{document}
