\documentclass{article}
\usepackage{geometry}
\usepackage[namelimits,sumlimits]{amsmath}
\usepackage{amssymb,amsfonts}
\usepackage{multicol}
\usepackage{mathrsfs}
\usepackage[cm]{fullpage}
\newcommand{\nc}{\newcommand}
\newcommand{\tab}{\hspace*{5em}}
\newcommand{\conj}{\overline}
\newcommand{\dd}{\partial}
\nc{\cn}{\mathbb{C}}
\nc{\rn}{\mathbb{R}}
\nc{\qn}{\mathbb{Q}}
\nc{\zn}{\mathbb{Z}}
\nc{\aff}{\mathbb{A}}
\nc{\proj}{\mathbb{P}}
\nc{\pd}[2]{\frac{\partial {#1}}{\partial {#2}}}
\nc{\ep}{\epsilon}
\nc{\topo}{\mathscr{T}}
\nc{\basis}{\mathscr{B}}
\nc{\nullset}{\varnothing}
\nc{\openm}{\begin{pmatrix}}
\nc{\closem}{\end{pmatrix}}
\DeclareMathOperator{\Gr}{Gr}
\begin{document}
Name: Hall Liu

Date: \today 
\vspace{1.5cm}

I'll probably be referencing Algebraic Geometry -- a First Course (Harris) a lot in this pset -- it's a lot more helpful than our textbook.
\subsection*{5.5.1}
If $V$ is a linear subvariety, then it is defined by $x_i=0$ for some set of $i$s after a linear change of variables. This is the intersection of a bunch of hypersurfaces defined by degree $1$ polynomials, so it has degree $1$.

Conversely, suppose $V$ has degree $1$ and dimension $d>0$. Further, suppose that $V$ is not a linear variety. Then, for some point $p\in V$, $V$ and $T_pV$ intersect with $\dim(V\cap T_pV)\leq 2d-n$ (I don't know how to formally prove this, but it seems to be true intuitively in low dimensions). Choose some subspace or superspace $W$ containing $p$ of $T_pV$ of dimension $n-d$ such that $W$ intersects $V$ at a finite number of points (again, no formal proof, but it seems right). Due to the definition of degree, $W\cap V$ consists of only the point $p$. However, $W$ and $V$ do not intersect transversely, since the dimension of the span of the tangent spaces of $W$ and $V$ is $\max(\dim T_pV, \dim W)$ since we have $T_pW=W$ and either $T_pV\subset W$ or $W\subset T_pV$. Thus, the intersection multiplicity at $p$ must be greater than $1$. However, by Thm 18.4 from Harris, $\deg W\cdot\deg V=1$ is the sum of the intersection multiplicities at all the subvarieties on which they intersect, but we showed that this must be greater than $1$. Thus, $V$ being nonlinear is a contradiction.
\subsection*{5.5.3}
Consider the conics $V(x^2+y^2-z^2)$ and $V(xy)$. Since these are both hypersurfaces formed from degree $2$ polynomials, they both have degree $2$. The example on page 94 shows that $V(x^2+y^2-z^2)$ is smooth, but if we examine the Jacobian of $xy$ at $[0:0:1]$, it's zero, so it has rank zero, which means that the dimension of the tangent space there is $2$, which disagrees with the dimension of the variety. Therefore they're not isomorphic.
\subsection*{5.6.1}
Let $W=v_d(V)$ and $p\in k[W]_n$, a degree $n$ polynomial. There exists a homomorphism from $k[W]_n$ to $k[V]_{nd}$ that's formed by taking the pullback $p\circ v_d$ of the Veronese map, so $p$ composed with the degree $d$ monomials from the Veronese map gives a degree $nd$ polynomial in $k[V]$. This map is injective, since if $p\circ v_d$ vanishes on $V$, then $p$ must vanish on the image of $V$, or $k[W]$, so $p$ is zero. This map is surjective, since we can express any degree $nd$ monomial as an appropriate product of the Veronese coefficients (i.e. $x_i^ax_j^b$ where $a+b=d$, proof later). Thus, the dimensions of $k[W]_n$ and $k[V]_{nd}$ are equal, so the Hilbert polynomial of $W$ is $P(nd)$, where $P$ is the Hilbert polynomial of $V$.

Proof of the monomial thingy: Let the monomial be $x_1^{j_1}\cdots x_k^{j_k}$, sum of the $j_i$ is $nd$. First, factor out all the $x_i^d$s, leaving behind only powers of $x_i$ less than $d$ and which powers sum up to a multiple of $d$. Then, match up all the powers of $x_1$ with a corresponding number of powers of $x_2$ (so that they add up to $d$) and remove them as a factor, do the same thing with the remaining powers of $x_2$ with $x_3$, and so on. This is guaranteed to leave nothing behind since we started with a multiple of $d$ total powers and are removing $d$ every time.
%Consider some $g\in k[V]$ of degree $n$. The inverse of the Veronese map induces a homomorphism from $k[V]$ to $k[W]$, so the image of $g$ is $g\circ F$, where $F$ is the inverse of the Veronese map. This is an element of $k[W]_n$, since $F$ is a projection. However, we can also view $g\circ F$ as an element of $k[V]_{nd}$ by ``loading in'' the degree $d$ monomials (in the coordinates of $V$) that are the coordinates of $W$. 
\subsection*{5}
a. The point $x$ is contained in some open affine subset of $\proj^3$, and since lines are determined by their restriction to some open affine subset, we can instead consider the set of lines passing through some point $x\in\aff^3$ instead. By translating $x$ to the origin, we see that this set of lines is in fact $\Gr(1,3)=\proj^2$. To show that it's a subvariety of $\Gr(2,4)$, note that lines in $\proj^3$ are synonymous with dimension $2$ subspaces of $\aff^4$ by taking the affine cone, and the requirement that a line pass through $p$ translates to the plane containing some line through the origin in $\aff^4$. This means that to span the plane, we need to fix one vector $p$ and pick some other $v$ not linearly dependent with $p$, so the image under the Plucker embedding is the set of products $p\wedge v$. Since the image of $\Gr(2,4)$ under the embedding is all products of the form $v_1\wedge v_2$, this is the condition that $v_1=p$, and it's a polynomial condition because if we expand it out in terms of some basis of $\bigwedge^2 \aff^4$, we get a polynomial in each coordinate of the coordinates of $v_1,v_2$.

b. Moving into $\aff^4$ again, the planes in $\proj^3$ correspond to $3$-dimensional subspaces of $\aff^2$ and the lines contained therein correspond to affine planes in those $3$-dimensional subspaces. Since we can associate $2$-d subspaces of $\aff^3$ to $1$-d subspaces by taking the orthogonal complement, this again corresponds to $\proj^2$. Now, using the argument above, instead of the plane in $\aff^4$ having to contain some line, we now need the plane to be orthogonal to some line $l$ determined by the orthogonal complement of the copy of $\aff^3$ that the plane $P\subset\proj^3$ corresponds to. If we let $l$ be the set of multiples of some vector $v$, this is the requirement (using the notation from above) that $v_1\cdot v=0$ and $v_2\cdot v=0$, which is once again a polynomial condition.
\subsection*{6}
a. If $\dim X=0$, then either $|X|=0$ or $1$, or $X$ is reducible but its only irreducible components are points. Thus, we want to show that projective varieties must only contain a finite number of irreducible components. For the sake of contradiction, suppose $X$ can only be written as the union of infinitely many irreducible components. Consider the set of subvarieties of $X$ that also have this property partially ordered under inclusion. It has a minimal element $Y$ due to Zorn's lemma because every descending chain terminates due to the Noetherian condition on homogeneous ideals. $Y$ is reducible, so we can write it as $Y_1\cup Y_2$, but both $Y_1$ and $Y_2$ can be written as the finite union of irreducible components due to minimality of $Y$. Thus such a minimal element does not exist, and $X$ must be finite.

If $X$ is finite, then any of its subvarieties besides the single point is reducible, so the chain of irreducible subvarieties has length $0$.

%b. Induct on $d$. For $d=0$, we have that $X$ is finite. Since there are an infinite number of disjoint $n-1$-dimensional linear subspaces of $\proj^n$, some of them must not intersect $X$, and any $n$-dimensional subspace of $\proj^n$ is the whole thing, which includes $X$. Now, let $\dim X=d$ and suppose it's true for all $k<d$. Then, for some hypersurface $V(f)$ not containing an irreducible component of $X$, there exists some $(n-d)$-dimensional plane $P$ that does not intersect $Y=X\cap V(f)$. Since we had a lot of freedom in choosing the hypersurface and the plane, it's probably safe to say that we can choose $f$ and the plane carefully enough that we get $|P\cap X|=\deg X$ (it's a bit difficult to make this formal). Since $\deg X$ is finite, we can use the argument from earlier to show that there's some $(n-d-1)$-dimensional plane that doesn't intersect $X$.
b. Suppose $\dim X=d$, and take some $(n-d)$-dimensional plane $P$ such that $\deg X=|X\cap P|$. We can do some coordinate changes so that these finite number of points all lie in the same affine patch. We have an infinite number of $(n-d-1)$-dimensional planes in $P$ that do not intersect within this patch, which implies that there exists some $(n-d-1)$-dimensional plane that doesn't hit $X$ at all. Conversely, ... something something
\subsection*{7}
The $k$-dim linear subspace is generated by the ideal $I=(f_1,\ldots,f_{n-k})$ where the $f_i$ are linearly independent homogeneous linear equations. We want to find the dimension of the $i$th grading of this ideal. Do a change of coordinates so that $f_i=x_i$. Then, the $i$th grading of the ideal has a basis consisting of the degree $i$ monomials in $n-k$ variables, of which there are $\binom{n-k+i-1}{i}$. Thus, the Hilbert function is the dimension of the $i$th grading of the whole ring minus this, or $\binom{n+i-1}{i}-\binom{n-k+i-1}{i}$.
\subsection*{8}
Examine the space $I_m$ of degree $m$ homogeneous polynomials. Consider a map $\psi:I_m\to k^d$, $f\mapsto (f(p_1),\ldots,f(p_d))$. We wish to show that this map is surjective, for then we would have that $\ker(\psi)$ is the space of polynomials which vanish on the $d$ points $p_1,\ldots,p_d$, and it'll have degree $\binom{m+n-1}{m}-d$, so subtracting this from $\binom{m+n-1}{m}$ would give $d$.

To show that the map is surjective, we need to show that for each $i$ we can find some $f$ such that $f(p_i)=1$ and $f(p_j)=0$ for $j\neq i$. A point in $\proj^n$ corresponds to a line through the origin in $\aff^{n+1}$, which is defined by the homogeneous linear equation $\sum a_ix_i$. Thus, to make the $f(p_j)$ be zero, we can take a polynomial that's the product of the polynomials corresponding to the $p_j$. This will have degree $n-1\leq m$. Then, we can just pad it up with extra linear factors which don't vanish at $p_i$ to get the desired $f$.
\end{document}
